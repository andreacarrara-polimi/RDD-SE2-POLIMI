Universities focus on academic subjects and emphasize theoretical knowledge, often leaving students without sufficient practical experience.
Internships fill this gap by helping students refine their skills, understand leadership dynamics, gain insights into workplace culture and build professional networks.
However, finding internships can be challenging due to limited knowledge of relevant companies and matching the right student with the right opportunity is even harder for both students and companies, given the difficulty in assessing competencies.
In this context, Students\&Companies comes to help by connecting university students with companies offering internships.

\section{Purpose}
Students\&Companies is designed to connect university students seeking internships with companies offering them.
The platform enables students to proactively search through a wide range of internship opportunities, filtering based on skills and interests.
It also provides personalized recommendations using advanced mechanisms, from keyword matching to statistical analysis.

Students can enhance their profiles by adding CVs and specifying job preferences, ensuring their applications are tailored and relevant.
Companies, in turn, can post detailed job descriptions, outlining project details, required skills and terms of the internship, such as compensation or mentorship opportunities, to attract the right candidates.

Beyond matching, S\&C streamlines the entire selection process.
Companies can manage interviews, use structured questionnaires to evaluate candidates and finalize decisions directly through the platform. 

S\&C also supports internships after placement.
Both students and companies can track progress, exchange comments through feedback tools and address concerns via complaint mechanisms, promoting transparency and accountability.
Universities can also actively oversee students' internships, ensuring quality experiences and adequate support.

Lastly, to feed the statistical analysis applied during recommendations, S\&C collects feedback about internships through forms from both students and companies, enhancing the platform's ability to refine and personalize its recommendation system.

\subsection{Goals}
By analyzing the previous description of the product and by clustering its features, the following goals emerge:

\renewcommand{\arraystretch}{1.5}
\begin{longtable}{|c|p{12.5cm}|}
    \hline \rowcolor{polimiblue!40}
    \textbf{ID} & \textbf{Description} \\ \hline
    G1 & S\&C helps match university students and internships based on students' profiles and internships' descriptions. \\ \hline
    G2 & S\&C allows students to search for internships and companies to search for students. \\ \hline
    G3 & S\&C recommends internships to students and students to companies. \\ \hline
    G4 & S\&C establishes a contact between matched students and companies. \\ \hline
    G5 & S\&C provides support for the selection process. \\ \hline
    G7 & S\&C lets students, companies and universities monitor the matchmaking process, the selection process and the internship's progress. \\ \hline
    G6 & S\&C collects feedback to improve recommendations. \\ \hline
\caption{Goals}
\end{longtable}

\section{Scope}
To clearly understand the scope of Students\&Companies, it is helpful to characterize its context using the world and the machine model \cite{jackson1995}.
The system is divided in two: the machine, representing the software to be developed, and the world, encompassing the portion of reality it influences.
This approach allows for a formal definition of the world and shared phenomena, facilitating a better analysis of the product's domain.

\subsection{World Phenomena}
The machine does not observe the following world phenomena:

\renewcommand{\arraystretch}{1.5}
\begin{longtable}{|c|p{10.5cm}|}
    \hline \rowcolor{polimiblue!40}
    \textbf{ID} & \textbf{Description} \\ \hline
    WP1 & A US wants to do an IN. \\ \hline
    WP2 & A CO wants to offer an IN. \\ \hline
    WP3 & A US writes his CV. \\ \hline
    WP4 & A US does an interview at a CO. \\ \hline
    WP5 & A US does an IN at a CO. \\ \hline
\caption{World phenomena}
\end{longtable}

\subsection{Shared Phenomena}
The following are phenomena involving both the world and the machine, categorized by their controller and observer:

\subsubsection{Controller: World - Observer : Machine}
\renewcommand{\arraystretch}{1.5}

\begin{longtable}{|c|p{10.5cm}|}
    \hline \rowcolor{polimiblue!40}
    \textbf{ID} & \textbf{Description} \\ \hline
    SP1 & A US registers on S\&C. \\ \hline
    SP2 & A US deletes his profile.\\ \hline
    SP3 & A US logs in S\&C. \\ \hline
    SP4 & A US logs out. \\ \hline
    SP5 & A US adds his CV. \\ \hline
    SP6 & A US updates his CV. \\ \hline
    SP7 & A US adds preferences on INs. \\ \hline
    SP8 & A US updates preferences on INs. \\ \hline
    SP9 & A US proactively searches for an IN. \\ \hline
    SP10 & A US applies to an IN. \\ \hline
    SP11 & A US withdraws an application. \\ \hline
    SP12 & A US views a recommended IN. \\ \hline
    SP13 & A US accepts a recommended IN. \\ \hline
    SP14 & A US fills out a feedback form. \\ \hline
    SP15 & A US views the status of a selection process. \\ \hline
    SP16 & A US updates the status of a selection process. \\ \hline
    SP17 & A US fills out a questionnaire for a selection process. \\ \hline
    SP18 & A US views the outcome of a selection process. \\ \hline
    SP19 & A US views the status of an IN. \\ \hline
    SP20 & A US updates the status of an IN. \\ \hline
    SP21 & A US comments an IN. \\ \hline
    SP22 & A US views a comment. \\ \hline

    SP23 & A CO registers on S\&C. \\ \hline
    SP24 & A CO deletes its profile. \\ \hline
    SP25 & A CO updates its profile. \\ \hline
    SP26 & A CO logs in S\&C. \\ \hline
    SP27 & A CO logs out. \\ \hline
    SP28 & A CO offers an IN. \\ \hline
    SP29 & A CO removes an IN. \\ \hline
    SP30 & A CO adds the description of an IN. \\ \hline
    SP31 & A CO updates the description of an IN. \\ \hline
    SP32 & A CO views a recommended US. \\ \hline
    SP33 & A CO accepts a recommended US. \\ \hline
    SP34 & A CO fills out a feedback form. \\ \hline
    SP35 & A CO views the status of a selection process. \\ \hline
    SP36 & A CO updates the status of a selection process. \\ \hline
    SP37 & A CO adds a questionnaire for a selection process. \\ \hline
    SP38 & A CO views a questionnaire for a selection process. \\ \hline
    SP39 & A CO updates the outcome of a selection process. \\ \hline
    SP40 & A CO views the status of an IN. \\ \hline
    SP41 & A CO updates the status of an IN. \\ \hline
    SP42 & A CO comments an IN. \\ \hline
    SP43 & A CO views a comment. \\ \hline
    
    SP44 & A UN registers on S\&C. \\ \hline
    SP45 & A UN deletes its profile. \\ \hline
    SP46 & A UN logs in S\&C. \\ \hline
    SP47 & A UN logs out. \\ \hline
    SP48 & A UN views the status of an IN. \\ \hline
    SP49 & A UN updates the status of an IN. \\ \hline
    SP50 & A UN comments an IN. \\ \hline
    SP51 & A UN views a comment. \\ \hline
\caption{Shared phenomena controlled by the world and observed by the machine}
\end{longtable}

\subsubsection{Controller: Machine - Observer : World}
\renewcommand{\arraystretch}{1.5}
\begin{longtable}{|c|p{10.5cm}|}
    \hline \rowcolor{polimiblue!40}
    \textbf{ID} & \textbf{Description} \\ \hline
    SP52 & S\&C notifies a US of a recommended IN. \\ \hline
    SP53 & S\&C notifies a CO of a recommended US. \\ \hline
    SP55 & S\&C notifies a US of an established contact with a CO. \\ \hline
    SP55 & S\&C notifies a CO of an established contact with a US. \\ \hline
    SP56 & S\&C notifies a US of an update on a selection process. \\ \hline
    SP57 & S\&C notifies a CO of an update on a selection process. \\ \hline
    SP58 & S\&C notifies a US if a CO adds a questionnaire for a selection process. \\ \hline
    SP59 & S\&C notifies a CO if a US filled out a questionnaire for a selection process. \\ \hline
    SP60 & S\&C notifies a US of an update on the status of an IN. \\ \hline
    SP61 & S\&C notifies a CO of an update on the status of an IN. \\ \hline
    SP62 & S\&C notifies a UN of an update on the status of an IN. \\ \hline
    SP63 & S\&C notifies a US of a comment on an IN. \\ \hline
    SP64 & S\&C notifies a CO of a comment on an IN. \\ \hline
    SP65 & S\&C notifies a UN of a comment on an IN. \\ \hline
\caption{Shared phenomena controlled by the machine and observed by the world}
\end{longtable}

\section{Glossary}
Below are listed definitions, acronyms and abbreviations used throughout the document:

\renewcommand{\arraystretch}{1.5}
\begin{longtable}{|c|p{8.5cm}|}
    \hline \rowcolor{polimiblue!40}
    \textbf{ID} & \textbf{Description} \\ \hline
    S\&C & Students\&Companies \\ \hline
    US & University student \\ \hline
    IN & Internship \\ \hline
    CO & Company \\ \hline
    UN & University \\ \hline
    CV & Curriculum vitae \\ \hline
\caption{Glossary}
\end{longtable}

\section{Reference Documents}
The document is structured according to the IEEE 29148:2018 standard for requirements engineering \cite{ieee2018} and the project assignment by Professor Di Nitto \cite{project2024}.

\section{Document Structure}
The purpose of the document is to analyze the requirements and describe the specification of Students\&Companies.
It is intended for developers tasked with implementation, but it also serves as a mutual agreement between customers and contractors.
Its structure follows a top-down approach, starting with a high-level overview of the problem and requirements and progressing to more detailed descriptions.
The document is divided into five chapters, each with a distinct scope and objective, as outlined below.

\subsubsection{Introduction}
The first chapter starts by contextualizing the problem, introducing the product and defining its goals.
It also provides a detailed list of global and shared phenomena and concludes with a glossary essential for understanding the subsequent chapters.

\subsubsection{Overall Description}
The second chapter offers a thorough examination of the problem, exploring the domain in greater detail through various scenarios and the use of class and state diagrams.
It also addresses product and user characteristics, along with assumptions and constraints.

\subsubsection{Specific Requirements}
The third chapter focuses on a detailed analysis of the specific requirements, providing comprehensive insights into external interface requirements, functional requirements and performance requirements.

\subsubsection{Formal Analysis}
The fourth chapter utilizes Alloy for formal analysis, aiming to validate the accuracy of the model outlined in the previous sections.
It emphasizes presenting the results of the checks performed and the relevant assertions.

\subsubsection{Effort Spent}
Chapter five quantifies the contributions of the two authors in writing this document.
