Universities focus on academic subjects and emphasize theoretical knowledge, often leaving students without sufficient practical experience.
Internships fill this gap by helping students refine their skills, understand leadership dynamics, gain insights into workplace culture and build professional networks.
However, finding internships can be challenging due to limited knowledge of relevant companies, and matching the right student with the right opportunity is even harder for both students and companies, given the difficulty in assessing competencies.
In this context, Students\&Companies comes to help by connecting university students and companies offering internships.

\section{Purpose}
Students\&Companies is designed to connect university students seeking internships with companies offering them.
The platform enables students to proactively search through a wide range of internship opportunities and it also provides personalized recommendations using advanced mechanisms such as statistical analysis.
Similarly, companies receive suggestions for students who stand out as strong candidates.

Students can enhance their profiles by adding CVs and specifying job preferences, ensuring their applications are tailored and relevant.
Companies, in turn, can post precise job descriptions, outlining project details, required skills and terms of the internship, such as compensation or mentorship opportunities.

Beyond matching, S\&C streamlines the entire selection process.
Companies can schedule interviews, use structured questionnaires to evaluate candidates and finalize decisions directly through the platform. 

The system also supports internships after placement.
Both students and companies can track progress and address concerns by exchanging comments, promoting transparency and accountability.
Universities can also actively oversee their students' internships, ensuring quality experiences and adequate support.

Lastly, to feed the statistical analysis used for recommendations, S\&C collects feedback about the platform through forms from both students and companies, enhancing its ability to refine and personalize the recommendation system.

\subsection{Goals}
By analyzing the previous description of the product and by clustering its features, the following goals emerge:

\newcounter{g}
\setcounter{g}{1}
\newcommand{\gc}{\theg\stepcounter{g}}
\renewcommand{\arraystretch}{1.5}
\begin{longtable}{|c|p{12.5cm}|}
    \hline \rowcolor{polimiblue!40}
    \textbf{ID} & \textbf{Description} \\ \hline
    G\gc & S\&C helps match university students and internships based on students' profiles and internships' descriptions. \\ \hline
    G\gc & S\&C allows students to search for internships. \\ \hline
    G\gc & S\&C recommends internships to students and students to companies. \\ \hline
    G\gc & S\&C establishes a contact between matched students and companies. \\ \hline
    G\gc & S\&C provides support for the selection process. \\ \hline
    G\gc & S\&C allows students, companies and universities monitor the matchmaking process, the selection process and the internship's progress. \\ \hline
    G\gc & S\&C collects feedback to improve recommendations. \\ \hline
\caption{Goals}
\end{longtable}

\section{Scope}
To clearly understand the scope of Students\&Companies, it is helpful to characterize its context using the world and the machine model \cite{jackson1995}.
The system is divided in two: the machine, representing the software to be developed, and the world, encompassing the portion of reality it influences.
This approach allows for a formal definition of the world and shared phenomena, facilitating a better analysis of the product's domain.

\subsection{World Phenomena}
The machine does not observe the following world phenomena:

\newcounter{wp}
\setcounter{wp}{1}
\newcommand{\wpc}{\thewp\stepcounter{wp}}
\renewcommand{\arraystretch}{1.5}
\begin{longtable}{|c|p{10.5cm}|}
    \hline \rowcolor{polimiblue!40}
    \textbf{ID} & \textbf{Description} \\ \hline
    WP\wpc & A US wants to do an IN. \\ \hline
    WP\wpc & A CO wants to host an IN. \\ \hline
    WP\wpc & A UN wants to monitor a US. \\ \hline
    WP\wpc & A US writes his CV. \\ \hline
    WP\wpc & A CO interviews a US. \\ \hline
    WP\wpc & A CO evaluates a US. \\ \hline
    WP\wpc & A US does an IN at a CO. \\ \hline
\caption{World phenomena}
\end{longtable}

\subsection{Shared Phenomena}
The following are phenomena involving both the world and the machine, categorized by their controller and observer:

\subsubsection{Controller: World - Observer: Machine}
\newcounter{sp}
\setcounter{sp}{1}
\newcommand{\spc}{\thesp\stepcounter{sp}}
\renewcommand{\arraystretch}{1.5}
\begin{longtable}{|c|p{10.5cm}|}
    \hline \rowcolor{polimiblue!40}
    \textbf{ID} & \textbf{Description} \\ \hline
    SP\spc & A US signs up on S\&C. \\ \hline
    SP\spc & A US deletes his profile.\\ \hline
    SP\spc & A US logs in S\&C. \\ \hline
    SP\spc & A US logs out. \\ \hline
    SP\spc & A US adds his CV. \\ \hline
    SP\spc & A US updates his CV. \\ \hline
    SP\spc & A US adds preferences on INs. \\ \hline
    SP\spc & A US updates preferences on INs. \\ \hline
    SP\spc & A US searches for an IN. \\ \hline
    SP\spc & A US views an IN. \\ \hline
    SP\spc & A US views a CO. \\ \hline
    SP\spc & A US applies for an IN. \\ \hline
    SP\spc & A US withdraws an application. \\ \hline
    SP\spc & A US views a recommended IN. \\ \hline
    SP\spc & A US accepts a recommended IN. \\ \hline
    SP\spc & A US declines a recommended IN. \\ \hline
    SP\spc & A US fills out a feedback form. \\ \hline
    SP\spc & A US views the status of a selection process. \\ \hline
    SP\spc & A US updates the status of a selection process. \\ \hline
    SP\spc & A US views a questionnaire for a selection process. \\ \hline
    SP\spc & A US fills out a questionnaire for a selection process. \\ \hline
    SP\spc & A US accepts a scheduled interview for a selection process. \\ \hline
    SP\spc & A US declines a scheduled interview for a selection process. \\ \hline
    SP\spc & A US views the outcome of a selection process. \\ \hline
    SP\spc & A US views the status of an IN. \\ \hline
    SP\spc & A US updates the status of an IN. \\ \hline
    SP\spc & A US comments an IN. \\ \hline
    SP\spc & A US views a comment. \\ \hline

    SP\spc & A CO signs up on S\&C. \\ \hline
    SP\spc & A CO deletes its profile. \\ \hline
    SP\spc & A CO logs in S\&C. \\ \hline
    SP\spc & A CO logs out. \\ \hline
    SP\spc & A CO updates its profile. \\ \hline
    SP\spc & A CO posts an IN. \\ \hline
    SP\spc & A CO removes an IN. \\ \hline
    SP\spc & A CO adds the description of an IN. \\ \hline
    SP\spc & A CO updates the description of an IN. \\ \hline
    SP\spc & A CO views a recommended US. \\ \hline
    SP\spc & A CO accepts a recommended US. \\ \hline
    SP\spc & A CO declines a recommended US. \\ \hline
    SP\spc & A CO fills out a feedback form. \\ \hline
    SP\spc & A CO views the status of a selection process. \\ \hline
    SP\spc & A CO updates the status of a selection process. \\ \hline
    SP\spc & A CO adds a questionnaire for a selection process. \\ \hline
    SP\spc & A CO views a questionnaire for a selection process. \\ \hline
    SP\spc & A CO schedules an interview for a selection process. \\ \hline
    SP\spc & A CO updates the outcome of a selection process. \\ \hline
    SP\spc & A CO views the status of an IN. \\ \hline
    SP\spc & A CO updates the status of an IN. \\ \hline
    SP\spc & A CO comments an IN. \\ \hline
    SP\spc & A CO views a comment. \\ \hline
    
    SP\spc & A UN signs up on S\&C. \\ \hline
    SP\spc & A UN deletes its profile. \\ \hline
    SP\spc & A UN logs in S\&C. \\ \hline
    SP\spc & A UN logs out. \\ \hline
    SP\spc & A UN views a US. \\ \hline
    SP\spc & A UN views an IN. \\ \hline
    SP\spc & A UN views a CO. \\ \hline
    SP\spc & A UN views the status of an IN. \\ \hline
    SP\spc & A UN updates the status of an IN. \\ \hline
    SP\spc & A UN comments an IN. \\ \hline
    SP\spc & A UN views a comment. \\ \hline
\caption{Shared phenomena controlled by the world and observed by the machine}
\end{longtable}

\subsubsection{Controller: Machine - Observer: World}
\renewcommand{\arraystretch}{1.5}
\begin{longtable}{|c|p{10.5cm}|}
    \hline \rowcolor{polimiblue!40}
    \textbf{ID} & \textbf{Description} \\ \hline
    SP\spc & S\&C notifies a US of a recommended IN. \\ \hline
    SP\spc & S\&C notifies a CO of a recommended US. \\ \hline
    SP\spc & S\&C notifies a US of a match with a CO. \\ \hline
    SP\spc & S\&C notifies a CO of a match with a US. \\ \hline
    SP\spc & S\&C notifies a US of an update on the status of a selection process. \\ \hline
    SP\spc & S\&C notifies a CO of an update on the status of a selection process. \\ \hline
    SP\spc & S\&C notifies a UN of an update on the status of a selection process. \\ \hline
    SP\spc & S\&C notifies a US if a CO adds a questionnaire for a selection process. \\ \hline
    SP\spc & S\&C notifies a CO if a US fills out a questionnaire for a selection process. \\ \hline
    SP\spc & S\&C notifies a US if a CO schedules an interview for a selection process. \\ \hline
    SP\spc & S\&C notifies a CO if a US accepts a scheduled interview for a selection process. \\ \hline
    SP\spc & S\&C notifies a CO if a US declines a scheduled interview for a selection process. \\ \hline
    SP\spc & S\&C notifies a US of the outcome of a selection process. \\ \hline
    SP\spc & S\&C notifies a US of an update on the status of an IN. \\ \hline
    SP\spc & S\&C notifies a CO of an update on the status of an IN. \\ \hline
    SP\spc & S\&C notifies a UN of an update on the status of an IN. \\ \hline
    SP\spc & S\&C notifies a US of a comment on an IN. \\ \hline
    SP\spc & S\&C notifies a CO of a comment on an IN. \\ \hline
    SP\spc & S\&C notifies a UN of a comment on an IN. \\ \hline
\caption{Shared phenomena controlled by the machine and observed by the world}
\end{longtable}

\section{Glossary}
Below are listed definitions, acronyms and abbreviations used throughout the document:

\renewcommand{\arraystretch}{1.5}
\begin{longtable}{|c|p{8.5cm}|}
    \hline \rowcolor{polimiblue!40}
    \textbf{ID} & \textbf{Description} \\ \hline
    S\&C & Students\&Companies \\ \hline
    US & University student \\ \hline
    IN & Internship \\ \hline
    CO & Company \\ \hline
    UN & University \\ \hline
    CV & Curriculum vitae \\ \hline
    Match & A US and a CO declare interest in each other \\ \hline
    Contact & Selection process and internship monitoring \\ \hline
    G\texttt{n} & Goal number \texttt{n} \\ \hline
    WP\texttt{n} & World phenomenon number \texttt{n} \\ \hline
    SP\texttt{n} & Shared phenomenon number \texttt{n} \\ \hline
    S\texttt{n} & Scenario number \texttt{n} \\ \hline
    F\texttt{n} & Function number \texttt{n} \\ \hline
    DA\texttt{n} & Domain assumption number \texttt{n} \\ \hline
    C\texttt{n} & Constraint number \texttt{n} \\ \hline
    UC\texttt{n} & Use case number \texttt{n} \\ \hline
\caption{Glossary}
\end{longtable}

\section{Reference Documents}
The document is structured according to the IEEE 29148:2018 standard for requirements engineering \cite{ieee2018} and the project assignment by Professor Di Nitto \cite{dinitto2024}.

\section{Document Structure}
The purpose of the document is to analyze the requirements and describe the specification of Students\&Companies.
It is intended for developers tasked with implementation, but it also serves as a mutual agreement between customers and contractors.
Its structure follows a top-down approach, starting with a high-level overview of the problem and its requirements and progressing to more detailed descriptions.
The document is divided into five chapters, each with a distinct scope and objective, as outlined below.

\subsubsection{Introduction}
The first chapter starts by contextualizing the problem, introducing the product and defining its goals.
It also provides a detailed list of global and shared phenomena, and concludes with a glossary essential for understanding the subsequent chapters.

\subsubsection{Overall Description}
The second chapter offers a thorough examination of the problem, exploring the domain in greater detail through various scenarios and the use of class and state diagrams.
It also addresses product and user characteristics, along with assumptions and constraints.

\subsubsection{Specific Requirements}
The third chapter focuses on a detailed analysis of the specific requirements, providing comprehensive insights into external interface requirements, functional requirements and performance requirements.

\subsubsection{Formal Analysis}
The fourth chapter utilizes Alloy for formal analysis, aiming to validate the accuracy of the model outlined in the previous sections.
It emphasizes presenting the results of the checks performed and the relevant assertions.

\subsubsection{Workload}
Chapter five quantifies the contributions of the two authors in writing this document.
