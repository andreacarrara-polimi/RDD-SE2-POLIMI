This chapter focuses on a detailed analysis of the specific requirements, offering comprehensive insights into external interface requirements, functional requirements and performance requirements, together with design constraints.

\section{External Interface Requirements}
The external interface requirements define how the Students\&Companies platform interacts with its users and external systems to deliver its core functionalities.
The following interfaces are designed to prioritize accessibility, accommodating a diverse range of user needs while maintaining robust system performance.

\subsection{User Interfaces}
The S\&CP will feature a modern, accessible and responsive web interface designed to meet the needs of students, companies and universities.
A mobile-friendly version ensures accessibility across devices, boosting user engagement.

\subsubsection{Wireframes}
To illustrate the platform’s functionality and design, wireframes follow to offer a structured, simplified visual representation of the key interfaces.
These wireframes will focus on the layout and navigation flow, emphasizing how users will interact with the system while ensuring the design aligns with usability goals.

\subsection{Hardware Interfaces}
The platform is designed to be hardware-agnostic, supporting all standard internet-enabled devices, including desktops, laptops, smartphones and tablets.
While the platform ensures compatibility with smaller devices, such as smartphones, features requiring detailed input, for example creating a questionnaire or uploading  documents, are better suited to a desktop or laptop environment for optimal user experience.

\subsection{Communication Interfaces}
The platform will rely on the HTTPS protocol to ensure secure data transmission, protecting user information during interactions.
SMTP will handle email notifications, ensuring timely delivery of updates and alerts.
Additionally, the system integrates with WebSocket to enable features such as live notifications and status updates for applications and recommendations.

\section{Functional Requirements}
\subsection{Use cases}
\subsection{Requirements}
\subsection{Mapping on goals}
\section{Performance Requirements}
To ensure an efficient and interactive experience, the Students\&Companies platform must meet stringent performance requirements that address scalability, data management and responsiveness under varying loads.

\subsubsection{Number of Concurrent Users}
Students\&Companies is expected to handle a significant user base, as it will cater to multiple universities.
To ensure a seamless experience, the platform must support at least 20\% of active users simultaneously, which could mean up to 50,000 users at peak times.
This guarantees stability and reliability during critical periods, such as application deadlines and recruitment drives.

\subsubsection{Data Storage}
The system must maintain records of both student profiles and company offerings.
Historical data such as past internships, feedback scores and selection outcomes should be preserved for analytics and future reference.
In general, no data should be deleted without explicit consent, ensuring transparency and trust.

\subsubsection{Response Time}
Core operations like user authentication, profile updates and search queries should execute within two seconds under normal conditions.
During high-load scenarios, response times may increase but must remain reasonable.

Additionally, the platform's matching algorithm should generate internship recommendations within three seconds, ensuring an friendly experience for users and avoiding their frustration and disengagement.
Feedback mechanisms must operate swiftly, delivering results to stakeholders within 10 seconds after computation.

\section{Design Constraints}
This section outlines the constraints that define the operational boundaries and compliance requirements of the Students\&Companies platform.

\subsection{Standard compliance}
Before using S\&C, all users must explicitly accept the platform's privacy policy.
The S\&CP must comply fully with GDPR regulations, ensuring transparency in how data is collected, stored and processed.
This includes providing users with tools to manage their personal data, such as viewing, editing or deleting their information.
Secure handling is particularly important for documents like CVs, which may contain highly sensitive personal information.

\subsection{Hardware limitations}
Users are required to access the S\&CP via devices equipped with stable internet connections and modern web browsers, mainly Chrome, Safari and Edge.
While desktops and laptops provide the optimal user experience for document uploads and extensive browsing, the platform must ensure functionality on mobile devices.
Additionally, the platform must account for varying device capabilities, including lower bandwidth environments, by employing lightweight designs and caching where appropriate.

\section{Software System Attributes}
This section describes the essential software qualities that the Students\&Companies platform must maintain to deliver a reliable, secure and interactive experience, capable of supporting its users effectively while accommodating future growth and enhancements.

\subsection{Reliability}
The system must be fault tolerant, capable of preventing error propagation and ensuring continuous usability.
Mechanisms such as database replication, automated failover and regular backups must be in place to maintain reliability.
Critical tasks like applications, postings and updates must remain operational even during system failures, with immediate recovery protocols.

\subsection{Availability}
The S\&CP must achieve a minimum availability rate of 99.9\%, limiting downtime to approximately 8.76 hours per year.
Maintenance should be minimized and scheduled during low-traffic periods, such as nighttime, to prevent disruptions.
Special care must be taken to ensure platform availability during peak usage periods, such as application deadlines or high-profile internship postings.

\subsection{Security}
The platform must implement robust access control, ensuring authentication to verify user identities and authorization to confirm permissions for specific actions.
All communication must be encrypted using protocols like HTTPS with TLS 1.3 or higher to prevent data breaches.
The database must employ measures to defend against vulnerabilities such as SQL injection attacks and all user credentials and personal data must be securely stored using encryption techniques.
Regular security audits, penetration testing and monitoring systems will ensure the platform's resilience against cyber threats.

\subsection{Maintainability}
The S\&CP must be designed using scalable and reusable models, allowing the addition of new features or improvements with minimal effort.
Modular architecture is essential to enable independent updates to components without impacting the entire system.
Maintenance windows must be scheduled during off-peak hours, typically at night, to ensure uninterrupted access during high-traffic periods.
Detailed documentation and automated testing must support maintenance and future development.

\subsection{Portability}
The platform must be accessible from all major web browsers, including Chrome, Safari and Edge, on both desktop and mobile devices.
The design must be responsive, ensuring usability across various screen sizes and resolutions.
While there are no specific server-side requirements, the system should support infrastructure migrations if necessary, such as moving to cloud-based solutions for enhanced scalability and flexibility.
