This chapter focuses on a detailed analysis of the specific requirements, offering comprehensive insights into external interface requirements, functional requirements and performance requirements, together with design constraints.

\section{External Interface Requirements}
The external interface requirements define how the Students\&Companies platform interacts with its users and external systems to deliver its core functionalities.
The following interfaces are designed to prioritize accessibility, accommodating a diverse range of user needs while maintaining robust system performance.

\subsection{User Interfaces}
The S\&C platform will feature a modern, accessible and responsive web interface designed to meet the needs of students, companies and universities.
A mobile-friendly version ensures accessibility across devices, boosting user engagement.

\subsubsection{Wireframe}
To illustrate the platform’s functionality and design, an exemplary wireframe follow to offer a structured, simplified visual representation of the key interface.

\subsection{Hardware Interfaces}
The platform is designed to be hardware-agnostic, supporting all standard internet-enabled devices, including desktops, laptops, smartphones and tablets.
While the platform ensures compatibility with smaller devices, such as smartphones, features requiring detailed input, for example creating a questionnaire or uploading  documents, are better suited to a desktop or laptop environment for optimal user experience.

\subsection{Communication Interfaces}
The platform will rely on the HTTPS protocol to ensure secure data transmission, protecting user information during interactions.
SMTP will handle email notifications, ensuring timely delivery of updates and alerts.
Additionally, the system integrates with WebSocket to enable features such as live notifications and status updates for applications and recommendations.

\section{Functional Requirements}
\subsection{Use Cases}


In this section ,they are explained and represented the main identified use cases through a table and a sequence diagram. Each table shows entry conditions, event flow, exit conditions and exceptions, meanwhile the sequence diagrams show the messages exchanged between the entities and the called functions. 

\newcounter{uc}
\setcounter{uc}{1}
\newcommand{\cuc}{\theuc\stepcounter{uc}}
\subsubsection*{UC\cuc . Register as a Student}
\begin{center}
    \begin{longtable}{|l|p{0.65\linewidth}|}
        \hline
        \textbf{Actor}            & US, email provider\\
        \hline
        \textbf{Entry conditions} & The US is not already registered in S\&C and has to search the S\&C URL in the browser search bar \\
        \hline
        \textbf{Event Flow}       & 1 - S\&C shows the login form.  \\
        & 2 - The US clicks on “Register as a Student” button.   \\
        & 3 - S\&C shows the registration form.    \\
        & 4 - The US inserts his name, surname, email, password and confirm password in the form.  \\
        & 5 - The US can decide to receive notification via email ticking on 'Keep Me Posted'.  \\
        & 6 - The US can decide to upload his CV and add personal preferences.  \\
        & 7 - The US clicks on the “Register” button.  \\
        & 8 - S\&C checks all the credentials.  \\
        & 9 - If credentials are correct S\&C sends a confirmation email to the US through the email provider.  \\
        & 10 - The S\&CP displays a Success Message (email sent).    \\
        & 11 - The US clicks on the confirmation link.    \\
        & 12 - The S\&CP redirects the US to the login form.    \\
        \hline
        \textbf{Exit condition}   & The US account is created and S\&C shows him the login form. \\       
        \hline
        \textbf{Exceptions}       & \begin{itemize}
            \item The email address is already linked to an account.
            \item The email address is not a valid institutional email. 
            \item The US data aren't valid.
            \item Invalid password if it is shorter than 8 characters, if it doesn’t have at least 1 number and/or 1 capital letter and/or a special character.
            \item 'Password' and 'Confirm Password' don't match.
        \end{itemize}
         In all cases the S\&CP will display an error message on the screen. 
        \\
        \hline
        \caption{Register as US use case.}
        \label{tab: Register_as_US_use_case}
    \end{longtable}
\end{center}
Sequence diagram here

\subsubsection*{UC\cuc . Register as a Company}
\begin{center}
    \begin{longtable}{|l|p{0.65\linewidth}|}
        \hline
        \textbf{Actor}            & CO, email provider \\
        \hline
        \textbf{Entry conditions} & The CO is not already registered in S\&C and has to search the S\&C URL in the browser search bar \\
        \hline
        \textbf{Event Flow}       & 1 - S\&C shows the login form.  \\
        & 2 - The CO clicks on “Register as a Company” button.   \\
        & 3 - S\&C shows the registration form.    \\
        & 4 - The CO inserts its name, email, VAT number, password and confirm password in the form.  \\
        & 5 - The CO can decide to insert a description and keywords about the company's area of expertise.  \\
        & 6 - The CO can decide to receive notification via email ticking on 'Keep Me Posted'\\
        & 7 - The CO clicks on the “Register” button.  \\
        & 8 - S\&C checks all the credentials.  \\
        & 9 - If credentials are correct S\&C sends a confirmation email to the CO through the email provider.  \\
        & 10 - The S\&CP displays a Success Message (email sent).    \\
        & 11 - The CO clicks on the confirmation link.    \\
        & 13 - The S\&CP redirects the CO to the login form.    \\
        \hline
        \textbf{Exit condition}   & The CO account is created and S\&C shows it the login form. \\
        \hline
        \textbf{Exceptions}       & \begin{itemize}
            \item The email address is already linked to an account. 
            \item Invalid password if it is shorter than 8 characters, if it doesn’t have at least 1 number and/or 1 capital letter and/or a special character. 
            \item 'Password' and 'Confirm Password' don't match.
            \item The CO data aren't valid.
        \end{itemize} In all cases the S\&CP will display an error message on the screen.  \\
        \hline
        \caption{Register as CO use case.}
        \label{tab: register_as_CO_use_case}
    \end{longtable}
\end{center}
Sequence diagram here

\subsubsection*{UC\cuc . Register as a University}
\begin{center}
    \begin{longtable}{|l|p{0.65\linewidth}|}
        \hline
        \textbf{Actor}            & UNI, email provider \\
        \hline
        \textbf{Entry conditions} & The UNI is not already registered in S\&C and has to search the S\&C URL in the browser search bar \\
        \hline
        \textbf{Event Flow}       & 1 - S\&C shows the login form.  \\
        & 2 - The UNI clicks on “Register as a University” button.   \\
        & 3 - S\&C shows the registration form.    \\
        & 4 - The UNI inserts its name, email, location password and confirm password in the form.  \\
        & 5 - The UNI can decide to receive notification via email ticking on 'Keep Me Posted'\\
        & 6 - The UNI clicks on the “Register” button.  \\
        & 7 - S\&C checks all the credentials.  \\
        & 8 - If credentials are correct S\&C sends a confirmation email to the UNI through the email provider.  \\
        & 9 - The S\&CP displays a Success Message (email sent).    \\
        & 10 - The UNI clicks on the confirmation link.    \\
        & 11 - The S\&CP redirects the University to the login form.    \\ 
        \hline
        \textbf{Exit condition}   & The UNI account is created and S\&C shows it the login form. \\
        \hline
        \textbf{Exceptions}       & \begin{itemize}
            \item The email address is already linked to an account. 
            \item Invalid password if it is shorter than 8 characters, if it doesn’t have at least 1 number and/or 1 capital letter and/or a special character. 
            \item 'Password' and 'Confirm Password' don't match.
            \item The UNI data aren't valid.
        \end{itemize} In all cases the S\&CP will display an error message on the screen.  
        \\
        \hline
        \caption{Register as UNI use case.}
        \label{tab: register_as_UNI_use_case}
    \end{longtable}
\end{center}

Sequence diagram here



\subsubsection*{UC\cuc . Login}
\begin{center}
    \begin{longtable}{|l|p{0.65\linewidth}|}
        \hline
        \textbf{Actor}            & Users\\
        \hline
        \textbf{Entry conditions} & The User is already registered in S\&C and has to search the S\&C URL in the browser search bar \\
        \hline
        \textbf{Event Flow}       & 1-\ S\&C shows the login form.  \\
        & 2-\ The User inserts his email and password in the form.\\
        & 3-\ The User clicks on the "Login" button.       \\              
        & 4-\ S\&C checks the credentials.  \\
        & 5-\ If the credentials are correct the User is redirected to the home page.  \\
        \hline
        \textbf{Exit condition}   & S\&C allows the User to access to the S\&C system and shows him the home page. \\
        \hline
        \textbf{Exceptions}       & Incorrect email or password. An error message is shown and the User is redirected back to the Login page. \\
        \hline
        \caption{Login use case.}
        \label{tab: login_use_case}
    \end{longtable}
\end{center}

Sequence diagram here

\subsubsection*{UC\cuc . Update personal information}
\begin{center}
    \begin{longtable}{|l|p{0.65\linewidth}|}
        \hline
        \textbf{Actor}            & US \\
        \hline
        \textbf{Entry conditions} & The US logged in, on the home page and he wants to update his personal information.\\
        \hline
        \textbf{Event Flow}       & 1-\ US clicks on 'My Profile' button on his home page. \\
        & 2-\ S\&C redirects the US to his profile page. \\
        & 3-\ US clicks on 'Change Personal Data' button.\\
        & 4-\ US makes changes.\\
        & 5-\ US clicks on 'Save Changes'.\\
        & 6-\ The S\&C displays a Success Message (data updated correctly).\\
        & 7-\ The US is redirected to his profile page.\\
        \hline
        \textbf{Exit condition}   &  The changes are correctly applied and the US is redirected to his profile page. \\
        \hline
        \textbf{Exceptions}       & \begin{itemize}
            \item The US changes his email and the email is not valid or already in use. 
            \item The US changes his password and it is shorter than 8 characters or if it doesn’t have at least 1 number and/or 1 capital letter and/or a special character. 
            \item The CV uploaded has a wrong format. 
        \end{itemize} In all cases S\&C will display an error message on the screen.  \\
        \hline
        \caption{US updates his personal information use case.}
        \label{tab: pi_use_case}
    \end{longtable}
\end{center}

\subsubsection*{UC\cuc . Acceptance of recommendation}
\begin{center}
    \begin{longtable}{|l|p{0.65\linewidth}|}
        \hline
        \textbf{Actor}            & US, email provider \\
        \hline
        \textbf{Entry conditions} & The US is registered and uploaded information about his preferences, his CV and selected 'Keep Me Posted' during registration. \\
        \hline
        \textbf{Event Flow}       & 1-\  US receives a mail notification about a job position suitable for his preferences. \\
        & 2-\ US clicks on the link in the email to view the job position.\\
        & 3-\ S\&C redirects the student to the position page.\\
        & 4-\ US clicks on 'Accept' button.\\
        & 5-\ The S\&C displays a Success Message (recommendation accepted)\\
        & 6-\ The US is redirected to his home page.\\
        \hline
        \textbf{Exit condition}   &  The recommendation is correctly accepted and the US is redirected to his home page. \\
        \hline
        \textbf{Exceptions}       & \begin{itemize}
            \item The position no longer exists.
            \item The recommendation process has already started.
            \item The CO that created the position deleted its profile.
        \end{itemize} In all cases the S\&C will display and error message and the US will be redirected to his home page. \\
        \hline
        \caption{US accepts a recommendation use case.}
        \label{tab: us_recommendation_use_case}
    \end{longtable}
\end{center}

\subsubsection*{UC\cuc . Withdrawal of application}
\begin{center}
    \begin{longtable}{|l|p{0.65\linewidth}|}
        \hline
        \textbf{Actor}            & US\\
        \hline
        \textbf{Entry conditions} & The US is logged on the S\&CP, he applied for an internship position and he's on the home page. \\
        \hline
        \textbf{Event Flow} & 1-\  The US clicks on 'My Applications' button\\
        & 2-\ S\&C redirects the US to the page showing the job positions he applied to. \\
        & 3-\ The US clicks on the job position he wants to withdraw from.\\
        & 4-\ The US clicks on 'Withdraw Application' button \\
        & 5-\ S\&C shows a confirm message.\\
        & 6-\ The US is redirect to his applications' page.\\
        & 7-\ If the selection process has already begun, S\&C notifies the company that posted the position about the candidate's withdrawal.\\
        \hline
        \textbf{Exit condition}   &  The user is no longer an applicant for the position and the position is removed from his applications \\
        \hline
        \textbf{Exceptions} &  \\
        \hline
        \caption{US withdraws an application.}
        \label{tab: us_recommendation_use_case}
    \end{longtable}
\end{center}

\subsubsection*{UC\cuc . Completion of a questionnaire}
\begin{center}
    \begin{longtable}{|l|p{0.65\linewidth}|}
        \hline
        \textbf{Actor} & US \\
        \hline
        \textbf{Entry conditions} & The US is currently logged in and has received an email notification stating that the company has uploaded a questionnaire for the position he applied to. The US wants to complete the questionnaire. The US is on his home page.\\
        \hline
        \textbf{Event Flow} & 1-\ US clicks on 'Questionnaires' button. \\
        & 2-\ S\&CP redirects the user to a page displaying the questionnaires he completed and those still pending.\\
        & 3-\ The US clicks on the questionnaire he wants to complete.\\
        & 4-\ S\&C shows the US the questionnaire.\\
        & 5-\ After completion the US clicks on 'Submit' button. \\
        & 6-\ S\&C shows a confirm message (questionnaire completed).\\
        & 7-\ The US is redirect to his questionnaires' page.\\
        & 8-\ S\&C informs the company that uploaded the questionnaire about the updated application status.\\
        \hline
        \textbf{Exit condition} &  The questionnaire is submitted correctly and the user is redirected to his questionnaires' page. \\
        \hline
        \textbf{Exceptions} & \begin{itemize}
            \item The selection process has already ended. 
            \item The questionnaire has expired.
            \item The US had previously withdrawn his application.
            \item The CO no longer exists.
            \item The position no longer exists.
        \end{itemize} In all cases S\&C will display an error message and redirect the US to his home page.
        \\
        \hline
        \caption{US completes a questionnaire use case.}
        \label{tab: questionnaire_use_case}
    \end{longtable}
\end{center}



\subsubsection*{UC\cuc . Visualize a company's profile}
\begin{center}
    \begin{longtable}{|l|p{0.65\linewidth}|}
        \hline
        \textbf{Actor}            & US \\
        \hline
        \textbf{Entry conditions} & The US is logged on S\&C and has searched and found a company he wants to view. \\
        \hline
        \textbf{Event Flow}       & 1-\ US clicks on the name of the company.  \\
        & 2-\ S\&CP redirects the US to the company's page.\\
        \hline
        \textbf{Exit condition}   &  The US is redirected to the company's page and can view its information.\\
        \hline
        \textbf{Exceptions}       & The company no longer exists. \\
        \hline
        \caption{US views a company's profile use case.}
        \label{tab: cp_use_case}
    \end{longtable}
\end{center}

\subsubsection*{UC\cuc . Search a company or a position with keywords }
\begin{center}
    \begin{longtable}{|l|p{0.65\linewidth}|}
        \hline
        \textbf{Actor}            & US \\
        \hline
        \textbf{Entry conditions} & The US is logged on S\&C and on the home page. \\
        \hline
        \textbf{Event Flow}       & 1-\ US taps on the search bar, types a keyword and presses 'return'.  \\
        & 2-\ S\&C redirects the US to the results of his search.\\
        \hline
        \textbf{Exit condition}   &  The user is redirected to a list of results related to the keyword. \\
        \hline
        \textbf{Exceptions}       & No results found for the keyword. \\
        \hline
        \caption{US search with a keywork use case.}
        \label{tab: cp_use_case}
    \end{longtable}
\end{center}



\subsubsection*{UC\cuc . Apply for a position}
\begin{center}
    \begin{longtable}{|l|p{0.65\linewidth}|}
        \hline
        \textbf{Actor}            & US \\
        \hline
        \textbf{Entry conditions} & The US is logged in S\&C and on his home page.  \\
        \hline
        \textbf{Event Flow}       & 1-\ US uses the search bar and type 'AI internship'.  \\
        & 2-\ S\&CP redirects the US to the results page. \\
        & 3-\ US clicks on the job position he's interested in. \\
        & 4-\ S\&CP redirects the US to the job information page. \\
        & 5-\ US clicks on the 'Apply' button. \\
        & 6-\ S\&CP displays a Success Message (applied successfully).\\
        & 7-\ S\&CP sends a confirmation email to the US containing info about the position through the email provider. \\
        & 8-\ S\&CP notifies the company that posted the position about the US' application. \\
        \hline
        \textbf{Exit condition}   &  The user is redirected to his applications' page and successfully applied to the position. \\
        \hline
        \textbf{Exceptions}       & \begin{itemize}
            \item The US already applied for that position. 
            \item The position no longer exists.
        \end{itemize} In all cases S\&C will display an error message and redirect the US to his home page.
        \\
        \hline
        \caption{US applies for a position use case.}
        \label{tab: cp_use_case}
    \end{longtable}
\end{center}

\subsubsection*{UC\cuc . Share information about a position}
\begin{center}
    \begin{longtable}{|l|p{0.65\linewidth}|}
        \hline
        \textbf{Actor}            & US \\
        \hline
        \textbf{Entry conditions} & The US is logged on S\&C and he's on the home page and wants to leave a comment about an ongoing internship. \\
        \hline
        \textbf{Event Flow}       & 1-\ US clicks on the 'My Internships' button.  \\
        & 2-\ S\&C redirects the US to his internships' page.\\
        & 3-\ US clicks on the IN he wants to share a comment about. \\
        & 4-\ S\&C redirects the US to the IN's page. \\
        & 5-\ US clicks on 'Leave a comment' button. \\
        & 6-\ S\&C shows the US a blank form.\\
        & 7-\ The US writes a comment. \\
        & 8-\ The US clicks on 'Submit' button.  \\
        & 9-\ S\&C shows a confirm message.\\
        & 10-\ S\&C notifies the CO about the new comment.\\
        & 11-\ S\&C notifies the US' university about the new comment.\\
        \hline
        \textbf{Exit condition}   &  The user is logged out from the S\&CP. \\
        \hline
        \textbf{Exceptions}   \\
        \hline
        \caption{US shares information about a position use case.}
        \label{tab: cp_use_case}
    \end{longtable}
\end{center}

\subsubsection*{UC\cuc . Review on platform's functionalities}
\begin{center}
    \begin{longtable}{|l|p{0.65\linewidth}|}
        \hline
        \textbf{Actor}            & US \\
        \hline
        \textbf{Entry conditions} & The US is logged on S\&C and he's on the home page. \\
        \hline
        \textbf{Event Flow}       & 1-\ US clicks on 'Give a feedback' button.  \\
        & 2-\ S\&C shows the US a form in which he can choose the 'Topic' which can be recommendation process or selection process and a blank space for the comment itself. \\
        & 3-\ The US chooses the topic \\
        & 4-\ The US writes a comment\\
        & 5-\ The US clicks on 'Submit' button.  \\
        & 6-\ S\&C shows a confirm message.\\
        \hline
        \textbf{Exit condition}   &  The feedback is submitted correctly. \\
        \hline
        \textbf{Exceptions}       & The comment is empty, an error message is displayed. \\
        \hline
        \caption{US views a company's profile use case.}
        \label{tab: cp_use_case}
    \end{longtable}
\end{center}

\subsubsection*{UC\cuc . Visualize an IN position}
\begin{center}
    \begin{longtable}{|l|p{0.65\linewidth}|}
        \hline
        \textbf{Actor}            & US \\
        \hline
        \textbf{Entry conditions} & The US is logged on S\&C and has searched and found an IN position he wants to view. \\
        \hline
        \textbf{Event Flow}       & 1-\ US clicks on the name of the IN position.  \\
        & 2-\ S\&C redirects the IN's information page.\\
        \hline
        \textbf{Exit condition}   &  The US is redirected to the IN's page and can view its information.\\
        \hline
        \textbf{Exceptions}       & \begin{itemize}
            \item The position no longer exists.
        \end{itemize} An error message is displayed and the US is redirected to his home page.
        \\
        \hline
        \caption{US views an IN position page use case.}
        \label{tab: cp_use_case}
    \end{longtable}
\end{center}




\subsubsection*{UC\cuc . Publish a job position}
\begin{center}
    \begin{longtable}{|l|p{0.65\linewidth}|}
        \hline
        \textbf{Actor}            & CO \\
        \hline
        \textbf{Entry conditions} & The CO is logged in. \\
        \hline
        \textbf{Event Flow}       & 1-\ CO clicks on 'Publish a Position' button. \\
        & 2-\ S\&C redirects CO to the 'Upload Position' form. \\
        & 3-\ CO types a name for the position.\\
        & 4-\ CO types a description.\\
        & 5-\ CO specifies the projects it involves. \\
        & 6-\ CO specifies the benefits it provides. \\
        & 7-\ CO specifies the duration of the IN. \\
        & 8-\ CO selects a deadline for applications. \\
        & 9-\ CO can add keywords that involves the areas of interest. \\
        & 10-\ CO clicks on 'Submit Position' button.\\
        & 10-\ S\&C uploads the position on the CO profile. \\
        & 11-\ S\&C sends a notification to all US that can be a match for that position. \\
        & 12 -\ CO is redirected to its home page.\\
        \hline
        \textbf{Exit condition}   & The position is correctly uploaded and CO is redirected to its home page. \\
        \hline
        \textbf{Exceptions}       & \begin{itemize}
            \item  One of the form's field is empty.
            \item  The deadline for applications is in the past.
        \end{itemize}In all cases an error messages is shown and the CO is redirected back to the creation form. \\
        \hline
        \caption{CO publish a job position use case.}
        \label{tab: cp_use_case}
    \end{longtable}
\end{center}


\subsubsection*{UC\cuc . Publish a questionnaire}
\begin{center}
    \begin{longtable}{|l|p{0.65\linewidth}|}
        \hline
        \textbf{Actor}            & CO \\
        \hline
        \textbf{Entry conditions} & The CO is logged in, and in the home page. \\
        \hline
        \textbf{Event Flow}       & 1-\ CO clicks on 'My Positions' button.  \\
        & 2-\ S\&C redirects CO to a page listing all positions it published. \\
        & 3-\ CO clicks on the name of the position it wants to add a questionnaire to. \\
        & 4-\ S\&C redirects the CO to the position page.\\
        & 5-\ CO clicks on 'Questionnaires' button. \\
        & 6-\ S\&C redirects CO to a page listing all questionnaire uploaded for that position. \\
        & 7-\ CO clicks on 'Add a Questionnaire' \\
        & 8-\ S\&C shows CO two different formats for the questionnaire, one pre-formatted in which the US can add questions or exercises and one completely blank.\\
        & 9-\ CO selects one of the two and completes it.\\
        & 9-\ CO selects a deadline for completion.\\
        & 11-\ CO clicks on 'Create' button.\\
        & 10-\ S\&C shows a confirmation message.\\
        & 11-\ S\&C redirects the CO to the questionnaires' page.\\
        & 12-\ S\&C notifies all US candidates about the new questionnaire.\\
        \hline
        \textbf{Exit condition}   &  The CO correctly creates the questionnaire and it's redirected to the questionnaires' page. \\
        \hline
        \textbf{Exceptions}       & \begin{itemize}
            \item The selection process already ended.
            \item The selection process hasn't started yet.
            \item The form is empty.
        \end{itemize}.
        In all cases S\&C shows an error message and the CO is redirected back to the form.
        \\
        \hline
        \caption{CO uploads a questionnaire use case.}
        \label{tab: cp_use_case}
    \end{longtable}
\end{center}


\subsubsection*{UC\cuc . Deletion of an IN position}
\begin{center}
    \begin{longtable}{|l|p{0.65\linewidth}|}
        \hline
        \textbf{Actor}            & CO \\
        \hline
        \textbf{Entry conditions} & The user is registered, logged in and in the home page.\\
        \hline
        \textbf{Event Flow}       & 1-\ CO clicks on 'My Positions' buttons. \\
        & 2-\ S\&C redirects CO to a page listing all positions it published. \\
        & 3-\ CO clicks on the name of the position it wants to delete. \\
        & 4-\ S\&C redirects the CO to the position page.\\
        & 5-\ CO clicks on 'Delete Position' button. \\
        & 6-\ S\&C shows a success message. \\
        & 7-\ CO is redirected to its published positions. \\
        \hline
        \textbf{Exit condition}   &  The position is correctly deleted and CO is redirected to the page listing the published positions. \\
        \hline
        \textbf{Exceptions}       & Position already deleted so S\&C shows an error message and redirects the CO to the home page. \\
        \hline
        \caption{CO deletes an IN use case.}
        \label{tab: cp_use_case}
    \end{longtable}
\end{center}


\subsubsection*{UC\cuc . Change information about an IN position}
\begin{center}
    \begin{longtable}{|l|p{0.65\linewidth}|}
        \hline
        \textbf{Actor}            & CO \\
        \hline
        \textbf{Entry conditions} & The user is registered, logged in and in the home page.\\
        \hline
        \textbf{Event Flow}       & 1-\ CO clicks on 'My Positions' buttons. \\
        & 2-\ S\&C redirects CO to a page listing all positions it published. \\
        & 3-\ CO clicks on the name of the position it wants to change. \\
        & 4-\ S\&C redirects the CO to the position page.\\
        & 5-\ CO clicks on 'Change Information' button. \\
        & 6-\ S\&C shows the form completed at creation time. \\
        & 7-\ CO applies changes. \\
        & 8-\ CO clicks on 'Apply Changes' button. \\
        & 9-\ S\&C shows a confirm message. \\
        \hline
        \textbf{Exit condition}   &  The position is correctly modified and CO is redirected to the page listing the published positions. \\
        \hline
        \textbf{Exceptions}       & The description is empty. The application deadline is in the past. The selection process already started. The selection process already ended. In all cases an error messages is shown and the CO is redirected back to the creation form. \\
        \hline
        \caption{CO deletes an IN use case.}
        \label{tab: cp_use_case}
    \end{longtable}
\end{center}

\subsubsection*{UC\cuc . Share information about a position}
\begin{center}
    \begin{longtable}{|l|p{0.65\linewidth}|}
        \hline
        \textbf{Actor}            & CO \\
        \hline
        \hline
        \textbf{Entry conditions} & The CO is registered, logged on the S\&CP and in his home page. \\
        \hline
        \textbf{Event Flow}       & 1-\ US clicks on the 'My Internships' button.  \\
        & 2-\ S\&C redirects CO to a page listing all ongoing internships.\\
        & 3-\ CO clicks on the IN's name he wants to share a comment about. \\
        & 4-\ S\&C redirects the CO to the IN's page. \\
        & 5-\ CO clicks on 'Leave a comment' button. \\
        & 6-\ S\&C shows the CO a blank form.\\
        & 7-\ The CO writes a comment. \\
        & 8-\ The CO clicks on 'Submit' button.  \\
        & 9-\ S\&C shows a confirm message.\\
        & 10-\ S\&C notifies the US about the new comment.\\
        & 11-\ S\&C notifies the US' university about the new comment.\\
        \hline
        \textbf{Exceptions}       & No exceptions to handle. \\
        \hline
        \caption{CO share information about a position use case.}
        \label{tab: cp_use_case}
    \end{longtable}
\end{center}

\subsubsection*{UC\cuc . Acceptance of recommendation}
\begin{center}
    \begin{longtable}{|l|p{0.65\linewidth}|}
        \hline
        \textbf{Actor}            & CO, email provider \\
        \hline
        \textbf{Entry conditions} & The CO is registered,uploaded a job position and ticked on 'Keep Me Posted' during registration. \\
        \hline
        \textbf{Event Flow}       & 1-\  CO receives a mail notification about a student suitable for the job it posted. \\
        & 2-\ CO clicks on the link in the email to view the student's profile. \\
        & 3-\ S\&C redirects the CO to the student information page\\
        & 4-\ US clicks on 'Accept' button.\\
        & 5-\ The S\&C displays a Success Message (recommendation accepted)\\
        \hline
        \textbf{Exit condition}   &  The CO correctly accepted the recommendation and it is redirected to the home page. \\
        \hline
        \textbf{Exceptions}       & Student no longer exists. Position no longer exist. \\
        \hline
        \caption{CO accepts a recommendation use case.}
        \label{tab: us_recommendation_use_case}
    \end{longtable}
\end{center}

\subsubsection*{UC\cuc . Update personal information}
\begin{center}
    \begin{longtable}{|l|p{0.65\linewidth}|}
        \hline
        \textbf{Actor}            & CO \\
        \hline
        \textbf{Entry conditions} & The CO is registered, logged in and wants to update his personal information\\
        \hline
        \textbf{Event Flow}       & 1-\ CO clicks on 'My Profile' button on his home page. \\
        & 2-\ S\&C redirects the CO to its profile page. \\
        & 3-\ CO clicks on 'Change Personal Data' button.\\
        & 4-\ CO makes changes.\\
        & 5-\ CO clicks on 'Save Changes'.\\
        & 6-\ The S\&C displays a Success Message (data updated correctly).\\
        \hline
        \textbf{Exit condition}   &  The changes are correctly saved and the CO is redirected to its profile page. \\
        \hline
        \textbf{Exceptions}       & \begin{itemize}
            \item If the CO changes his email and the email is not valid or already in use. 
            \item If the CO changes his password and it is shorter than 8 characters or if it doesn’t have at least 1 number and/or 1 capital letter and/or a special character. 
            \item The description about the CO is empty. 
        \end{itemize} In all cases the S\&C will display an error message on the screen.  \\
        \hline
        \caption{CO updates his personal information use case.}
        \label{tab: pi_use_case}
    \end{longtable}
\end{center}

\subsubsection*{UC\cuc . Visualize a student's profile}
\begin{center}
    \begin{longtable}{|l|p{0.65\linewidth}|}
        \hline
        \textbf{Actor}            & US \\
        \hline
        \textbf{Entry conditions} & The CO is logged on S\&C and has searched and found a student it wants to view. \\
        \hline
        \textbf{Event Flow}       & 1-\ CO clicks on the name of the student.  \\
        & 2-\ S\&CP redirects the CO to the student's page.\\
        \hline
        \textbf{Exit condition}   &  The CO is redirected to the student's page and can view its information.\\
        \hline
        \textbf{Exceptions}       & The student no longer exists. \\
        \hline
        \caption{CO views a student's profile use case.}
        \label{tab: cp_use_case}
    \end{longtable}
\end{center}

\subsubsection*{UC\cuc . Search a student}
\begin{center}
    \begin{longtable}{|l|p{0.65\linewidth}|}
        \hline
        \textbf{Actor}            & US \\
        \hline
        \textbf{Entry conditions} & The CO is logged on S\&C and on the home page. \\
        \hline
        \textbf{Event Flow}       & 1-\ CO taps on the search bar, types a student's name and presses 'return'.  \\
        & 2-\ S\&C redirects the CO to the results of his search.\\
        \hline
        \textbf{Exit condition}   &  The CO is redirected to a list of results related to the keyword. \\
        \hline
        \textbf{Exceptions}       & No results found for that name. \\
        \hline
        \caption{CO search a student use case.}
        \label{tab: cp_use_case}
    \end{longtable}
\end{center}


\subsubsection*{UC\cuc . Review on platform's functionalities}
\begin{center}
    \begin{longtable}{|l|p{0.65\linewidth}|}
        \hline
        \textbf{Actor}            & CO \\
        \hline
        \textbf{Entry conditions} & The CO is logged in S\&C and it's on the home page. \\
        \hline
        \textbf{Event Flow}       & 1-\ CO clicks on 'Give a feedback' button on his home page.  \\
        & 2-\ S\&CP shows the CO a form in which he can write.\\
        & 3-\ CO can specify key words regarding the topic of the feedback.\\
        & 4-\ The CO writes a comment and clicks on 'Submit' button.\\
        & 5-\ S\&CP displays a Success Message (feedback submitted)\\
        \hline
        \textbf{Exit condition}   &  The feedback is submitted correctly. \\
        \hline
        \textbf{Exceptions}       & The comment is empty, an error message is displayed. \\
        \hline
        \caption{CO gives feedback use case.}
        \label{tab: cp_use_case}
    \end{longtable}
\end{center}


\subsubsection*{UC\cuc . Visualize a position}
\begin{center}
    \begin{longtable}{|l|p{0.65\linewidth}|}
        \hline
        \textbf{Actor}            & CO \\
        \hline
        \hline
        \textbf{Entry conditions} & The CO is logged in S\&C and in its home page. \\
        \hline
        \textbf{Event Flow}       & 1-\ US clicks on the 'My Positions' button.  \\
        & 2-\ S\&C redirects CO to a page listing all positions it published.\\
        & 3-\ CO clicks on a position name \\
        & 4-\ S\&CP redirects the CO to the position's page.\\
        \hline
        \textbf{Exit condition}   & CO is able to know specific features of the job application that it selected. \\
        \hline
        \hline
        \caption{CO views a job position use case.}
        \label{tab: cp_use_case}
    \end{longtable}
\end{center}

\subsubsection*{UC\cuc . Deletion of profile}
\begin{center}
    \begin{longtable}{|l|p{0.65\linewidth}|}
        \hline
        \textbf{Actor}            & Users \\
        \hline
        \textbf{Entry conditions} & The user is registered and logged on the S\&CP. \\
        \hline
        \textbf{Event Flow}       & 1-\ User clicks on 'My Profile' button.   \\
        & 2-\ S\&CP redirects user to his personal profile page.\\
        & 3-\ User clicks on 'Delete My Profile' button.\\
        & 4-\ S\&C sends a confirmation email to the user through the email provider with a deletion link.\\
        & 5-\ User clicks on link in the email. \\
        & 6-\ S\&CP displays a Success Message (profile deleted)\\
        & 7-\ The S\&CP redirects the user to the registration page.\\
        \hline
        \textbf{Exit condition}   &  S\&CP deletes user's profile and user is redirected to the platform home page. \\
        \hline
        \textbf{Exceptions}       & The profile was already deleted. \\
        \hline
        \caption{US views a company's profile use case.}
        \label{tab: cp_use_case}
    \end{longtable}
\end{center}


\subsubsection*{UC\cuc . Log out}
\begin{center}
    \begin{longtable}{|l|p{0.65\linewidth}|}
        \hline
        \textbf{Actor}            & Users \\
        \hline
        \textbf{Entry conditions} & The user is registered and logged in \\
        \hline
        \textbf{Event Flow}       & 1-\  The user presses the 'Log out' button. \\
        & 2-\ The S\&CP displays a Success Message.\\
        & 3-\ The S\&CP redirects the user to the main page of S\&CP.\\
        \hline
        \textbf{Exit condition}   &  The user is logged out from the S\&CP. \\
        \hline
        \textbf{Exceptions}       & No exceptions to handle. \\
        \hline
        \caption{Logout use case.}
        \label{tab: logout_use_case}
    \end{longtable}
\end{center}

\subsubsection*{UC\cuc. View student status}
\begin{center}
    \begin{longtable}{|l|p{0.65\linewidth}|}
        \hline
        \textbf{Actor}            & UNI \\
        \hline
        \textbf{Entry conditions} & The UNI is registered and has received a notification about a comment regarding an ongoing internship of one of its students.\\
        \hline
        \textbf{Event Flow}       & 1-\ UNI logs in S\&C \\
        & 2-\ UNI clicks on 'My Students' button. \\
        & 3-\ S\&C redirects the UNI to the page showing its students. \\
        & 4-\ UNI clicks on the search bar of the page to look for the student. \\
        & 5-\ S\&C shows the results.\\
        & 6-\ UNI clicks on the students' name.\\
        & 7-\ S\&C redirects the UNI to a page showing all comments regarding that student and the ongoing internship.\\
        \hline
        \textbf{Exit condition}   &  UNI sees the latest comment it received the notification about. \\
        \hline
        \textbf{Exceptions}       &  \\
        \hline
        \caption{UNI visualizes a US status use case.}
        \label{tab: cp_use_case}
    \end{longtable}
\end{center}

\subsubsection*{UC\cuc . Update personal information}
\begin{center}
    \begin{longtable}{|l|p{0.65\linewidth}|}
        \hline
        \textbf{Actor}            & UNI \\
        \hline
        \textbf{Entry conditions} & The UNI is registered and wants to update his personal information\\
        \hline
        \textbf{Event Flow}       & 1-\ UNI clicks on 'My Profile' button on his home page. \\
        & 2-\ S\&CP redirects UNI to his personal page. \\
        & 3-\ UNI clicks on 'Change Personal Data' button.\\
        & 4-\ UNI makes changes.\\
        & 5-\ The S\&CP displays a Success Message (data updated correctly).\\
        \hline
        \textbf{Exit condition}   &  The US is redirected to his personal page. \\
        \hline
        \textbf{Exceptions}       & \begin{itemize}
            \item If the UNI changes email and the email is not valid or already in use. 
            \item If the UNI changes the password and it is shorter than 8 characters or if it doesn’t have at least 1 number and/or 1 capital letter and/or a special character.  
        \end{itemize} In all cases the S\&CP will display an error message on the screen.  \\
        \hline
        \caption{UNI updates his personal information use case.}
        \label{tab: pi_use_case}
    \end{longtable}
\end{center}

\subsection{Requirements}
\renewcommand{\arraystretch}{1.5}
\begin{longtable}{|c|p{10.5cm}|}
    \hline \rowcolor{polimiblue!40}
    \textbf{ID} & \textbf{Description} \\ \hline
    R1 & S\&C allows USs to sign up. \\ \hline
    R2 & S\&C allows USs to delete their profile.\\ \hline
    R3 & S\&C allow USs to log in. \\ \hline
    R4 & S\&C allows USs to log out. \\ \hline
    R5 & S\&C allows USs to add their CV. \\ \hline
    R6 & S\&C allows USs to update their CV. \\ \hline
    R7 & S\&C allows USs to add preferences on INs. \\ \hline
    R8 & S\&C allows USs to update preferences on INs. \\ \hline
    R9 & S\&C allows USs to search for an IN. \\ \hline
    R10 & S\&C allows USs to view an IN. \\ \hline
    R11 & S\&C allows USs to view a CO. \\ \hline
    R12 & S\&C allows USs to apply to an IN. \\ \hline
    R13 & S\&C allows USs to withdraw an application. \\ \hline
    R14 & S\&C allows USs to view a recommended IN. \\ \hline
    R15 & S\&C allows USs to accept a recommended IN. \\ \hline
    R16 & S\&C allows USs to decline a recommended IN. \\ \hline
    R17 & S\&C allows USs to fill out a feedback form. \\ \hline
    R18 & S\&C allows USs to view the status of a selection process. \\ \hline
    R19 & S\&C allows USs to update the status of a selection process. \\ \hline
    R20 & S\&C allows USs to view a questionnaire for a selection process. \\ \hline
    R21 & S\&C allows USs to fill out a questionnaire for a selection process. \\ \hline
    R22 & S\&C allows USs to accept a scheduled interview for a selection process. \\ \hline
    R23 & S\&C allows USs to decline a scheduled interview for a selection process. \\ \hline
    R24 & S\&C allows USs to view the outcome of a selection process. \\ \hline
    R25 & S\&C allows USs to view the status of an IN. \\ \hline
    R26 & S\&C allows USs to update the status of an IN. \\ \hline
    R27 & S\&C allows USs to comment an IN. \\ \hline
    R28 & S\&C allows USs to view a comment. \\ \hline

    R29 & S\&C allows COs to sign up. \\ \hline
    R30 & S\&C allows COs to delete their profile. \\ \hline
    R31 & S\&C allows COs to log in. \\ \hline
    R32 & S\&C allows COs to log out. \\ \hline
    R33 & S\&C allows COs to update their profile. \\ \hline
    R34 & S\&C allows COs to offer an IN. \\ \hline
    R35 & S\&C allows COs to remove an IN. \\ \hline
    R36 & S\&C allows COs to add the description of an IN. \\ \hline
    R37 & S\&C allows COs to update the description of an IN. \\ \hline
    R38 & S\&C allows COs to view a recommended US. \\ \hline
    R39 & S\&C allows COs to accept a recommended US. \\ \hline
    R40 & S\&C allows COs to decline a recommended US. \\ \hline
    R41 & S\&C allows COs to fill out a feedback form. \\ \hline
    R42 & S\&C allows COs to view the status of a selection process. \\ \hline
    R43 & S\&C allows COs to update the status of a selection process. \\ \hline
    R44 & S\&C allows COs to add a questionnaire for a selection process. \\ \hline
    R45 & S\&C allows COs to view a questionnaire for a selection process. \\ \hline
    R46 & S\&C allows COs to schedule an interview for a selection process. \\ \hline
    R47 & S\&C allows COs to update the outcome of a selection process. \\ \hline
    R48 & S\&C allows COs to view the status of an IN. \\ \hline
    R49 & S\&C allows COs to update the status of an IN. \\ \hline
    R50 & S\&C allows COs to comment an IN. \\ \hline
    R51 & S\&C allows COs to view a comment. \\ \hline
    
    R52 & S\&C allows UNs to sign up on. \\ \hline
    R53 & S\&C allows UNs to delete their profile. \\ \hline
    R54 & S\&C allows UNs to log in. \\ \hline
    R55 & S\&C allows UNs to log out. \\ \hline
    R56 & S\&C allows UNs to view a US. \\ \hline
    R57 & S\&C allows UNs to view an IN. \\ \hline
    R58 & S\&C allows UNs to view a CO. \\ \hline
    R59 & S\&C allows UNs to view the status of an IN. \\ \hline
    R60 & S\&C allows UNs to update the status of an IN. \\ \hline
    R61 & S\&C allows UNs to comment an IN. \\ \hline
    R62 & S\&C allows UNs to view a comment. \\ \hline
    
    R63 & S\&C notifies a US of a recommended IN. \\ \hline
    R64 & S\&C notifies a CO of a recommended US. \\ \hline
    R65 & S\&C notifies a US of a match with a CO. \\ \hline
    R66 & S\&C notifies a CO of a match with a US. \\ \hline
    R67 & S\&C notifies a US of an update on a selection process. \\ \hline
    R68 & S\&C notifies a CO of an update on a selection process. \\ \hline
    R69 & S\&C notifies a US if a CO adds a questionnaire for a selection process. \\ \hline
    R70 & S\&C notifies a CO if a US fills out a questionnaire for a selection process. \\ \hline
    R71 & S\&C notifies a US if a CO schedules an interview for a selection process. \\ \hline
    R72 & S\&C notifies a CO if a US accepts a scheduled interview for a selection process. \\ \hline
    R73 & S\&C notifies a CO if a US declines a scheduled interview for a selection process. \\ \hline
    R74 & S\&C notifies a US of an update on the status of an IN. \\ \hline
    R75 & S\&C notifies a CO of an update on the status of an IN. \\ \hline
    R76 & S\&C notifies a UN of an update on the status of an IN. \\ \hline
    R77 & S\&C notifies a US of a comment on an IN. \\ \hline
    R78 & S\&C notifies a CO of a comment on an IN. \\ \hline
    R79 & S\&C notifies a UN of a comment on an IN. \\ \hline
\caption{Requirements}
\end{longtable}

\subsection{Mapping on Goals}
The following table presents the traceability matrix, demonstrating the alignment between Specific requirements and corresponding goals to ensure coverage and accountability.

\renewcommand{\arraystretch}{1.5}
\begin{longtable}{|c|c|c|c|c|c|c|c|}
    \hline \rowcolor{polimiblue!40}
    \textbf{Requirement} & \textbf{G1} & \textbf{G2} & \textbf{G3} & \textbf{G4} & \textbf{G5} & \textbf{G6} & \textbf{G7} \\ \hline
    R1 & 1 & 2 & 3 & 4 & 5 & 6 & 7 \\ \hline
    R2 & 1 & 2 & 3 & 4 & 5 & 6 & 7 \\ \hline
    R3 & 1 & 2 & 3 & 4 & 5 & 6 & 7 \\ \hline
    R4 & 1 & 2 & 3 & 4 & 5 & 6 & 7 \\ \hline
    R5 & 1 & 2 & 3 & 4 & 5 & 6 & 7 \\ \hline
    R6 & 1 & 2 & 3 & 4 & 5 & 6 & 7 \\ \hline
    R7 & 1 & 2 & 3 & 4 & 5 & 6 & 7 \\ \hline
    R8 & 1 & 2 & 3 & 4 & 5 & 6 & 7 \\ \hline
    R9 & 1 & 2 & 3 & 4 & 5 & 6 & 7 \\ \hline
    R10 & 1 & 2 & 3 & 4 & 5 & 6 & 7 \\ \hline
    R11 & 1 & 2 & 3 & 4 & 5 & 6 & 7 \\ \hline
    R12 & 1 & 2 & 3 & 4 & 5 & 6 & 7 \\ \hline
    R13 & 1 & 2 & 3 & 4 & 5 & 6 & 7 \\ \hline
    R14 & 1 & 2 & 3 & 4 & 5 & 6 & 7 \\ \hline
    R15 & 1 & 2 & 3 & 4 & 5 & 6 & 7 \\ \hline
    R16 & 1 & 2 & 3 & 4 & 5 & 6 & 7 \\ \hline
    R17 & 1 & 2 & 3 & 4 & 5 & 6 & 7 \\ \hline
    R18 & 1 & 2 & 3 & 4 & 5 & 6 & 7 \\ \hline
    R19 & 1 & 2 & 3 & 4 & 5 & 6 & 7 \\ \hline
    R20 & 1 & 2 & 3 & 4 & 5 & 6 & 7 \\ \hline
    R21 & 1 & 2 & 3 & 4 & 5 & 6 & 7 \\ \hline
    R22 & 1 & 2 & 3 & 4 & 5 & 6 & 7 \\ \hline
    R23 & 1 & 2 & 3 & 4 & 5 & 6 & 7 \\ \hline
    R24 & 1 & 2 & 3 & 4 & 5 & 6 & 7 \\ \hline
    R25 & 1 & 2 & 3 & 4 & 5 & 6 & 7 \\ \hline
    R26 & 1 & 2 & 3 & 4 & 5 & 6 & 7 \\ \hline
    R27 & 1 & 2 & 3 & 4 & 5 & 6 & 7 \\ \hline
    R28 & 1 & 2 & 3 & 4 & 5 & 6 & 7 \\ \hline
    R29 & 1 & 2 & 3 & 4 & 5 & 6 & 7 \\ \hline
    R30 & 1 & 2 & 3 & 4 & 5 & 6 & 7 \\ \hline
    R31 & 1 & 2 & 3 & 4 & 5 & 6 & 7 \\ \hline
    R32 & 1 & 2 & 3 & 4 & 5 & 6 & 7 \\ \hline
    R33 & 1 & 2 & 3 & 4 & 5 & 6 & 7 \\ \hline
    R34 & 1 & 2 & 3 & 4 & 5 & 6 & 7 \\ \hline
    R35 & 1 & 2 & 3 & 4 & 5 & 6 & 7 \\ \hline
    R36 & 1 & 2 & 3 & 4 & 5 & 6 & 7 \\ \hline
    R37 & 1 & 2 & 3 & 4 & 5 & 6 & 7 \\ \hline
    R38 & 1 & 2 & 3 & 4 & 5 & 6 & 7 \\ \hline
    R39 & 1 & 2 & 3 & 4 & 5 & 6 & 7 \\ \hline
    R40 & 1 & 2 & 3 & 4 & 5 & 6 & 7 \\ \hline
    R41 & 1 & 2 & 3 & 4 & 5 & 6 & 7 \\ \hline
    R42 & 1 & 2 & 3 & 4 & 5 & 6 & 7 \\ \hline
    R43 & 1 & 2 & 3 & 4 & 5 & 6 & 7 \\ \hline
    R44 & 1 & 2 & 3 & 4 & 5 & 6 & 7 \\ \hline
    R45 & 1 & 2 & 3 & 4 & 5 & 6 & 7 \\ \hline
    R46 & 1 & 2 & 3 & 4 & 5 & 6 & 7 \\ \hline
    R47 & 1 & 2 & 3 & 4 & 5 & 6 & 7 \\ \hline
    R48 & 1 & 2 & 3 & 4 & 5 & 6 & 7 \\ \hline
    R49 & 1 & 2 & 3 & 4 & 5 & 6 & 7 \\ \hline
    R50 & 1 & 2 & 3 & 4 & 5 & 6 & 7 \\ \hline
    R51 & 1 & 2 & 3 & 4 & 5 & 6 & 7 \\ \hline
    R52 & 1 & 2 & 3 & 4 & 5 & 6 & 7 \\ \hline
    R53 & 1 & 2 & 3 & 4 & 5 & 6 & 7 \\ \hline
    R54 & 1 & 2 & 3 & 4 & 5 & 6 & 7 \\ \hline
    R55 & 1 & 2 & 3 & 4 & 5 & 6 & 7 \\ \hline
    R56 & 1 & 2 & 3 & 4 & 5 & 6 & 7 \\ \hline
    R57 & 1 & 2 & 3 & 4 & 5 & 6 & 7 \\ \hline
    R58 & 1 & 2 & 3 & 4 & 5 & 6 & 7 \\ \hline
    R59 & 1 & 2 & 3 & 4 & 5 & 6 & 7 \\ \hline
    R60 & 1 & 2 & 3 & 4 & 5 & 6 & 7 \\ \hline
    R61 & 1 & 2 & 3 & 4 & 5 & 6 & 7 \\ \hline
    R62 & 1 & 2 & 3 & 4 & 5 & 6 & 7 \\ \hline
    R63 & 1 & 2 & 3 & 4 & 5 & 6 & 7 \\ \hline
    R64 & 1 & 2 & 3 & 4 & 5 & 6 & 7 \\ \hline
    R65 & 1 & 2 & 3 & 4 & 5 & 6 & 7 \\ \hline
    R66 & 1 & 2 & 3 & 4 & 5 & 6 & 7 \\ \hline
    R67 & 1 & 2 & 3 & 4 & 5 & 6 & 7 \\ \hline
    R68 & 1 & 2 & 3 & 4 & 5 & 6 & 7 \\ \hline
    R69 & 1 & 2 & 3 & 4 & 5 & 6 & 7 \\ \hline
    R70 & 1 & 2 & 3 & 4 & 5 & 6 & 7 \\ \hline
    R71 & 1 & 2 & 3 & 4 & 5 & 6 & 7 \\ \hline
    R72 & 1 & 2 & 3 & 4 & 5 & 6 & 7 \\ \hline
    R73 & 1 & 2 & 3 & 4 & 5 & 6 & 7 \\ \hline
    R74 & 1 & 2 & 3 & 4 & 5 & 6 & 7 \\ \hline
    R75 & 1 & 2 & 3 & 4 & 5 & 6 & 7 \\ \hline
    R76 & 1 & 2 & 3 & 4 & 5 & 6 & 7 \\ \hline
    R77 & 1 & 2 & 3 & 4 & 5 & 6 & 7 \\ \hline
    R78 & 1 & 2 & 3 & 4 & 5 & 6 & 7 \\ \hline
    R79 & 1 & 2 & 3 & 4 & 5 & 6 & 7 \\ \hline
\caption{Traceability matrix}
\end{longtable}

\section{Performance Requirements}
To ensure an efficient and interactive experience, the Students\&Companies platform must meet stringent performance requirements that address scalability, data management and responsiveness under varying loads.

\subsubsection{Number of Concurrent Users}
Students\&Companies is expected to handle a significant user base, as it will cater to multiple universities.
To ensure a seamless experience, the platform must support at least 20\% of active users simultaneously, which could mean up to 50,000 users at peak times.
This guarantees stability and reliability during critical periods, such as application deadlines and recruitment drives.

\subsubsection{Data Storage}
The system must maintain records of both student profiles and company offerings.
Historical data such as past internships, feedback scores and selection outcomes should be preserved for analytics and future reference.
In general, no data should be deleted without explicit consent, ensuring transparency and trust.

\subsubsection{Response Time}
Core operations like user authentication, profile updates and search queries should execute within two seconds under normal conditions.
During high-load scenarios, response times may increase but must remain reasonable.

Additionally, the platform's matching algorithm should generate internship recommendations within three seconds, ensuring an friendly experience for users and avoiding their frustration and disengagement.
Feedback mechanisms must operate swiftly, delivering results to stakeholders within 10 seconds after computation.

\section{Design Constraints}
This section outlines the constraints that define the operational boundaries and compliance requirements of the Students\&Companies platform.

\subsection{Standard Compliance}
Before using S\&C, all users must explicitly accept the platform's privacy policy.
The platform must comply fully with GDPR regulations, ensuring transparency in how data is collected, stored and processed.
This includes providing users with tools to manage their personal data, such as viewing, editing or deleting their information.
Secure handling is particularly important for documents like CVs, which may contain highly sensitive personal information.

\subsection{Hardware Limitations}
Users are required to access S\&C via devices equipped with stable internet connections and modern web browsers, mainly Chrome, Safari and Edge.
While desktops provide the optimal user experience for uploading document and adding questionnaires, the platform must ensure functionality on mobile devices.
Additionally, the platform must account for varying device capabilities, including lower bandwidth environments, by employing lightweight designs and caching where appropriate.

\section{Software System Attributes}
This section describes the essential software qualities that the Students\&Companies platform must maintain to deliver a reliable, secure and interactive experience, capable of supporting its users effectively while accommodating future growth and enhancements.

\subsection{Reliability}
The system must be fault tolerant, capable of preventing error propagation and ensuring continuous usability.
Mechanisms such as database replication, automated failover and regular backups must be in place to maintain reliability.
Critical tasks like applications, postings and updates must remain operational even during system failures, with immediate recovery protocols.

\subsection{Availability}
S\&C must achieve a minimum availability rate of 99.9\%, limiting downtime to approximately 8.76 hours per year.
Maintenance should be minimized and scheduled during low-traffic periods, such as nighttime, to prevent disruptions.
Special care must be taken to ensure platform availability during peak usage periods, such as application deadlines or high-profile internship postings.

\subsection{Security}
The platform must implement robust access control, ensuring authentication to verify user identities and authorization to confirm permissions for specific actions.
All communication must be encrypted using protocols like HTTPS with TLS 1.3 or higher to prevent data breaches.
The database must employ measures to defend against vulnerabilities such as SQL injection attacks and all user credentials and personal data must be securely stored using encryption techniques.
Regular security audits, penetration testing and monitoring systems will ensure the platform's resilience against cyber threats.

\subsection{Maintainability}
S\&C must be designed using scalable and reusable models, allowing the addition of new features or improvements with minimal effort.
A modular architecture is essential to enable independent updates to components without impacting the entire system.
Maintenance windows must be scheduled during off-peak hours, typically at night, to ensure uninterrupted access during high-traffic periods.
Detailed documentation and automated testing must support maintenance and future development.

\subsection{Portability}
The platform must be accessible from all major web browsers, including Chrome, Safari and Edge, on both desktop and mobile devices.
The design must be responsive, ensuring usability across various screen sizes and resolutions.
While there are no specific server-side requirements, the system should support infrastructure migrations if necessary, such as moving to cloud-based solutions for enhanced scalability and flexibility.
