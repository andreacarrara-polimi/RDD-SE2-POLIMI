This chapter provides a detailed description of the whole problem, offering a deeper exploration of the domain through various scenarios and the application of class and state diagrams.
It also covers key aspects such as product and user characteristics, as well as the underlying assumptions and constraints shaping the context.

\section{Product Perspective}
This section employs scenarios to illustrate the system's role, supported by class and state diagrams that define its structure and dynamic behavior.
These tools collectively establish a clear understanding of how the product integrates into its environment and addresses the identified problem.

\subsection{Scenarios}
\subsubsection{Scenario 1: Registration and Profile Setup}
AlgoSphere is a cutting-edge technology company specializing in algorithm development and artificial intelligence solutions for businesses across diverse industries.
To cultivate fresh talent and build a pipeline of future employees, AlgoSphere is considering offering several internship positions.
To save valuable time and resources, the company opts to use Students\&Companies.
It navigates to the S\&C platform via a browser, selects the option "Register as a Company"
and provides its name, email and field of operation.

Ben Pinter, a first-year Master's student in computer science, is eager to gain real-world experience in AI through an internship.
To streamline his search and avoid the inefficiency of navigating multiple websites and job boards, Ben registers on the S\&CP, hoping to find opportunities aligned with his academic background and career goals.
Accessing the platform via a browser, he selects the "Register as a Student" option and completes his profile by providing essential details, including his name, surname and email.
He also has the option to upload add information such as his CV and personal preferences.
These preferences may include availability, expected salary and benefits.

Lastly, TechVille University, where Ben is enrolled, recognizes the importance of staying involved in students’ professional development to ensure they gain valuable experience and meet academic standards.
To facilitate this, TechVille University registers on the S\&CP by selecting "Register as a University" and providing its details. By linking Ben’s profile to the university through his institutional email, the platform allows TechVille University to automatically monitor Ben’s internships, ensuring alignment with educational goals and providing support throughout his professional journey.

\subsubsection{Scenario 2: Internship Posting and Candidate Matching}
AlgoSphere wants to publish an internship position in AI. It clicks the corresponding button and describe the opportunity by outlining its tasks, application domains, project scopes and offered terms such as salary and benefits.
After publishing, the S\&CP recommends Ben’s profile on AlgoSphere’s homepage, as his skills and experience closely match the internship requirements.
In turn, the platform notifies Ben about the opportunity.

Meanwhile, John Kent, another first-year Master's student in computer science, is already registered on the platform but has not uploaded his CV or specified preferences.
Interested in AI systems for cybersecurity, John uses the search bar to look for relevant opportunities by entering "AI."
Among the results, he finds AlgoSphere’s internship and decides to apply, which prompts a notification to AlgoSphere.
To strengthen his application and improve his chances of being selected, John uploads his CV.

Upon reviewing John’s application, AlgoSphere determines that his profile merits consideration. The platform notifies both John and AlgoSphere of the established contact.

\subsubsection{Scenario 3: Selection Process}
After being notified of a matching, both Ben and AlgoSphere accept the S\&CP's recommendation, formally initiating contact and starting the selection process.

AlgoSphere uploads questionnaires tailored to the internship position, prompting notifications to both Ben and John to complete them. When one submits them, the platform notifies AlgoSphere of the update and allows the company to review the answers.

After evaluating the questionnaires, AlgoSphere selects Ben as the stronger candidate due to his more aligned skills and experience, finalizing the selection process. John is notified of his rejection, while Ben receives a notification detailing the next steps. The notification offers Ben the choice to schedule either a remote interview via the platform or an in-person interview, providing flexibility to move forward conveniently.

\subsubsection{Scenario 4: Monitoring Matchmaking, Selection and Internships}
The S\&CP promotes a transparent exchange of information between students and companies throughout the matchmaking process, the selection process and the ongoing internship.
Notifications ensure that all parties, including universities, remain informed and engaged.

For example, AlgoSphere can use the platform's dedicated forms to provide explanations for selecting or rejecting candidates, detailed progress reports or other insights.
Notifications keep Ben and his university informed of these updates in real time.

In turn, if Ben encounters an issue during his internship, such as unclear expectations from AlgoSphere, he comment a complaint through the platform's tool.
This complaint is immediately visible to both AlgoSphere and his university.

\subsubsection{Scenario 4: Feedback Forms}
Both Ben and AlgoSphere can provide feedback on the S\&CP regarding the matchmaking and selection processes, including the recommendations.
Ben can share comments with respect to the opportunities he was provided with, the efficiency of the selection process and the quality of the recommendations he received.
In the same way, companies can evaluate the platform's ability to provide suitable candidate matches, the effectiveness of its communication tools and the overall usability of the system during selection.

\subsection{Class Diagram}
\subsection{State Diagrams}
\section{Product Functions}
The following is a summary of the main functions of Students\&Companies, providing a concise abstraction and generalization of its goals and scenarios.
This serves as an intermediate step bridging goals, scenarios and use cases.

\textbf{\textcolor{polimiblue}{Function 1: Registering as a Company}}

A company registers by providing its name, email and field of operation.

\textbf{\textcolor{polimiblue}{Function 2: Registering as a Student}}

Students register by submitting their name, surname and institutional email.
They can upload a CV and set preferences.

\textbf{\textcolor{polimiblue}{Function 3: Registering as a University}}

Universities register by submitting their name and email.
Students with the same institutional email domain are automatically linked, allowing seamless monitoring.

\textbf{\textcolor{polimiblue}{Function 4: Posting an Internship}}

Companies can publish internship opportunities by specifying their tasks, application domains, project scopes and terms.

\textbf{\textcolor{polimiblue}{Function 5: Searching for Internships}}

Students use the search bar to find internships by keywords. 

\textbf{\textcolor{polimiblue}{Function 6: Recommendations}}

Students are recommended to companies based on skills and preferences. In turn, students are recommended tailored opportunities from suitable companies.

\textbf{\textcolor{polimiblue}{Function 7: Applying for an Internship}}

Students apply for internships and notifications are sent to the corresponding company.

\textbf{\textcolor{polimiblue}{Function 8: Application Management}}

Companies receive notifications for new applications and can review candidates’ profiles.

\textbf{\textcolor{polimiblue}{Function 9: Selection Process}}

Companies assess candidates through questionnaires and scheduled interviews, available in both virtual and in-person.
They then finalize the outcome, with notifications keeping both parties informed at every stage.

\textbf{\textcolor{polimiblue}{Function 10: Internship Monitoring}}

Students, companies and universities are updated of the status of an internship through notifications.
The platform also enables students to submit complaints through comments.

\textbf{\textcolor{polimiblue}{Function 11: Feedback Submission}}

Students and companies provide feedback on matchmaking to refine the statistical analysis on which recommendations are based.

\textbf{\textcolor{polimiblue}{Function 12: Logout}}

Users can end their session and must log in again to reaccess the platform.

\section{User Characteristics}
Students\&Companies has three types of users: university students, companies and universities.
Below is a brief explanation of their main characteristics and identities.

\subsubsection{University Students}
University students are the users seeking internships.
A student can register by providing a valid institutional email address and a password in their personal data.
They can also upload a CV and specify personal preferences, including availability and benefits.

\subsubsection{Companies}
Companies are the users offering internships and they are typically represented by the human resources department acting on behalf of the organization.
Registration requires providing the company name, a valid email address and its field of expertise.

\subsubsection{Universities}
Universities are the users monitoring their students and they are typically represented by the career services department acting on behalf of the institution.
Registration requires the university's name and a valid institutional email address.

\section{Assumptions and Constraints}
The table below outlines the domain assumptions of Students\&Companies:

\renewcommand{\arraystretch}{1.5}
\begin{longtable}{|c|p{12.5cm}|}
    \hline \rowcolor{polimiblue!40}
    \textbf{ID} & \textbf{Description} \\ \hline \hline
    DA1 & USs have a valid institutional email. \\ \hline
    DA2 & COs have a valid business email. \\ \hline
    DA3 & UNIs have a valid institutional email. \\ \hline
    DA4 & USs have a working device with a reliable internet connection. \\ \hline
    DA5 & COs have a working device with a reliable internet connection. \\ \hline
    DA6 & UNIs have a working device with a reliable internet connection. \\ \hline
\caption{Domain assumptions}
\end{longtable}

The following table lists the constraints of S\&C: 

\renewcommand{\arraystretch}{1.5}
\begin{longtable}{|c|p{12.5cm}|}
    \hline \rowcolor{polimiblue!40}
    \textbf{ID} & \textbf{Description} \\ \hline \hline
    C2 & The S\&CP cannot prevent a CO from offering an IN. \\ \hline
    C2 & The S\&CP cannot prevent a US from applying to an IN. \\ \hline
    C2 & A CO cannot prevent a US from applying to an IN. \\ \hline
    C1 & A UN cannot prevent a US from applying to an IN. \\ \hline
    C3 & The S\&CP cannot start a selection process before a match. \\ \hline
    C5 & The S\&CP cannot end a selection process before an outcome. \\ \hline
    C4 & The S\&CP cannot start an IN before the end of a selection process. \\ \hline
\caption{Constraints}
\end{longtable}
