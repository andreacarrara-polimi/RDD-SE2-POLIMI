\documentclass[a4paper, oneside]{book}

\usepackage{hyperref}
\usepackage{biblatex}

\bibliography{bibliography}

\title{
    Students\&Companies \\
    \Huge Requirements Analysis \\
    and Specification Document
}
\author{Andrea Carrara and Federica Currò Dossi}

\begin{document}

\maketitle

\tableofcontents

\chapter{Introduction}
Universities teach academic subjects and emphasize theoretical knowledge.
On the other hand, internships offer a crucial dimension thus often lacking in students, that of practical experience.
They help students refine their skills, understand leadership dynamics, gain insights into workplace cultures and build professional networks.
However, finding internships is hard, due to limited knowledge of relevant companies, and finding the right fit is even harder, both for students and companies, due to difficulties in assessing competencies.
In this context, Students\&Companies comes to help by matching university students and internships offered by companies.

\section{Purpose}
\subsection{Goals}
\begin{table}[h]
\renewcommand{\arraystretch}{1.5}
\begin{tabular}{|c|p{10.5 cm}|}
    \hline
    \textbf{ID} & \textbf{Description} \\ \hline \hline
    G1 & S\&C helps match university students and internships based on students' skills and internships' terms. \\ \hline
    G2 & S\&C allows students to search for internships and companies to search for students. \\ \hline
    G3 & S\&C recommends internships to students and students to companies. \\ \hline
    G4 & S\&C establishes a contact between matched students and companies. \\ \hline
    G5 & S\&C provides support for the selection process. \\ \hline
    G6 & S\&C collects data to improve recommendations. \\ \hline
    G7 & S\&C helps universities, students and companies monitor the matchmaking process, the selection process and the internship's status. \\ \hline
\end{tabular}
\caption{Goals.}
\end{table}

\section{Scope}
\subsection{World Phenomena}
\subsection{Shared Phenomena}
\section{Glossary}
\begin{itemize}
    \item[\textbf{S\&C}] Students\&Companies
\end{itemize}

\section{Reference Documents}
The document is structured according to the project assignment and the IEEE 29148:2018 \cite{ieee2018} standard for requirements engineering.

\section{Document Structure}

\chapter{Overall Description}
\section{Product Perspective}
\section{Product Functions}
\section{User Characteristics}
\section{Assumptions, Dependencies and Constraints}

\chapter{Specific Requirements}
\section{External Interface Requirements}
\section{Functional Requirements}
\section{Performance Requirements}
\subsection{Design Constraints}
\subsection{Software System Attributes}

\chapter{Formal Analysis}

\chapter{Workload}

\printbibliography

\end{document}
