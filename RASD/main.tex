\documentclass[a4paper, oneside]{book}

\title{
  Students \& Companies \\
  \Huge Requirements Analysis \\
  and Specification Document
}
\author{Andrea Carrara, Federica Currò Dossi}
\date{November - December 2024}

\begin{document}

\maketitle

\tableofcontents

\chapter{Introduction}
\section{Purpose}
The purpose of this document is to present a thorough description of {Students\&Companies}. It is intended for developers tasked with implementing the outlined requirements and it can be employed as a mutual agreement between the customers and the contractors. Additionally, it aims to give the customer a clear and unambiguous description of the system's features, functionalities and constraints, enabling the customer to validate the requirements and to verify if the system meets the expectations. 

\subsection{Goals}

\begin{itemize}
    \item [\textbf{G1}] S\&C helps match university students and internships based on students' skills and internships' terms.
    \item [\textbf{G2}] S\&C allows students to search for internships and companies to search for students.
    \item [\textbf{G3}] S\&C recommends internships to students and students to companies.
    \item [\textbf{G4}] S\&C establishes a contact between matched students and companies.
    \item [\textbf{G5}] S\&C provides support for the selection process.
    \item [\textbf{G6}] S\&C collects data to improve recommendations.
    \item [\textbf{G7}] S\&C helps universities, students and companies monitor the matchmaking process, the selection process and internships' status.
\end{itemize}

\section{Scope}
\subsection{World Phenomena}
\subsection{Shared Phenomena}
\section{Document Structure}

\chapter{Functionalities}
\section{User Perspective}
\subsection{Scenarios}
\section{Product Perspective}
\section{Assumptions}
\subsection{User Characteristics}
\subsection{Domain Assumptions}

\chapter{Requirements}
\section{Interface Requirements}
\section{Functional Requirements}
\subsection{Use Cases}
\subsection{Requirements}
\section{Non-Functional Requirements}
\subsection{Design Constraints}
\subsection{System Attributes}

\chapter{Formal Analysis}

\chapter{Workload}

% Glossary

% Bibliography

\end{document}
